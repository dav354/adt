\documentclass[12pt,oneside,a4paper,parskip=half]{scrbook}

% Sprachanpassung und Grundkonfiguration
\usepackage[utf8]{inputenc}
\usepackage[T1]{fontenc}
\usepackage[ngerman]{babel}
\usepackage{lmodern}
\usepackage{microtype}      % bessere Umbrüche
\usepackage{csquotes}
\usepackage{hyphenat}       % Zeilenumbruch in \texttt
\usepackage{newunicodechar}
\newunicodechar{ }{\,}      % geschütztes schmales Leerzeichen

% Seitenlayout
\usepackage[a4paper,left=20mm,right=20mm,top=20mm,bottom=25mm]{geometry}
\usepackage{setspace}
\onehalfspacing
\sloppy                     % großzügigere Zeilenumbrüche (vermeidet Overfull hboxes)

% Mathematik & Symbole
\usepackage{amsmath,amsfonts,amssymb}

% Tabellen & Grafiken
\usepackage{graphicx}
\usepackage{float}
\usepackage{booktabs}
\usepackage{longtable}
\usepackage{tabularx}
\usepackage{pdflscape}
\usepackage{placeins}
\graphicspath{{./}{./figures/}}

% Aufzählungen
\usepackage{enumitem}

% Code-Listings
\usepackage{xcolor}
\usepackage{listings}
\colorlet{punct}{red!60!black}
\definecolor{background}{HTML}{EEEEEE}
\definecolor{delim}{RGB}{20,105,176}
\colorlet{numb}{magenta!60!black}
\definecolor{pblue}{rgb}{0.13,0.13,1}
\definecolor{pgreen}{rgb}{0,0.5,0}
\definecolor{pred}{rgb}{0.9,0,0}
\definecolor{pgrey}{rgb}{0.46,0.45,0.48}

\lstdefinestyle{code}{
  basicstyle=\ttfamily,
  columns=fullflexible,
  showstringspaces=false,
  numbers=left,
  numberstyle=\scriptsize,
  stepnumber=1,
  numbersep=8pt,
  backgroundcolor=\color{background},
  commentstyle=\color{pgrey}\itshape,
  keywordstyle=\color{pblue},
  stringstyle=\color{pred},
  breaklines=true,
  breakatwhitespace=true,
  tabsize=2,
  keepspaces=true,
  linewidth=\textwidth
}

\lstdefinelanguage{json}{
  basicstyle=\normalfont\ttfamily,
  numbers=left,
  numberstyle=\scriptsize,
  stepnumber=1,
  numbersep=8pt,
  showstringspaces=false,
  breaklines=true,
  backgroundcolor=\color{background},
  literate=
   *{0}{{{\color{numb}0}}}{1}
    {1}{{{\color{numb}1}}}{1}
    {2}{{{\color{numb}2}}}{1}
    {3}{{{\color{numb}3}}}{1}
    {4}{{{\color{numb}4}}}{1}
    {5}{{{\color{numb}5}}}{1}
    {6}{{{\color{numb}6}}}{1}
    {7}{{{\color{numb}7}}}{1}
    {8}{{{\color{numb}8}}}{1}
    {9}{{{\color{numb}9}}}{1}
    {:}{{{\color{punct}{:}}}}{1}
    {,}{{{\color{punct}{,}}}}{1}
    {\{}{{{\color{delim}{\{}}}}{1}
    {\}}{{{\color{delim}{\}}}}}{1}
    {[}{{{\color{delim}{[}}}}{1}
    {]}{{{\color{delim}{]}}}}{1},
}

\lstdefinelanguage{xml}{
  morestring=",
  morestring={>}{<},
  morecomment={<?}{?>},
  stringstyle=\color{black},
  identifierstyle=\color{pblue},
  keywordstyle=\color{pgreen},
  morekeywords={xmlns,version,type}
}

\lstdefinelanguage{Java}{
  showspaces=false,
  showtabs=false,
  tabsize=4,
  breaklines=true,
  keepspaces=true,
  showstringspaces=false,
  commentstyle=\color{pgrey}\itshape,
  keywordstyle=\color{pblue},
  stringstyle=\color{pred},
  basicstyle=\ttfamily
}

\lstset{style=code}

%%%%%%%%%%%%%%%%%%%
%% definitions
%%%%%%%%%%%%%%%%%%%
\def\BaAuthor{Noah Raupold (5022097),\\ David Gläsle (5022114)}
\def\BaAuthorPDF{Noah Raupold (5022097), David Gläsle (5022114)}
\def\BaAuthorStudyProgram{Informatik}
\def\BaType{ADT Portfolio Teil 3}
\def\BaTitle{Implementierung des Lobbyregisters}
\def\BaDeadline{\today}

\def\iswithfullname{1}
\ifdefined\iswithfullname
  \def\ShowBaAuthor{\BaAuthor}
\else
  \def\ShowBaAuthor{N.~N.}
\fi

\newcommand{\TitleGraphic}{%
  \IfFileExists{Logo.png}{%
    \includegraphics[width=0.5\textwidth]{Logo.png}%
  }{%
    \fbox{\parbox[c][3cm][c]{0.5\textwidth}{Logo.png fehlt}}%
  }%
}

\newcommand*{\forcetwosidetitle}{
  \begingroup
  \cleardoubleoddpage
  \KOMAoptions{titlepage=true}%
  \csname @twosidetrue\endcsname
  \maketitle
  \endgroup
}

\newcommand{\TOCbreak}{\\}

% Bibliografie (Biber)
\usepackage[backend=biber,style=numeric]{biblatex}
\IfFileExists{literatur.bib}{\addbibresource{literatur.bib}}{}

% --- TikZ (nur bei Bedarf) ---
\usepackage{tikz}
\usetikzlibrary{arrows.meta,positioning,calc,fit,backgrounds}
\usepackage{adjustbox} % für max totalsize

% Hyperlinks zuletzt laden
\usepackage{hyperref}
\hypersetup{
  colorlinks=true,
  linkcolor=black,
  filecolor=magenta,
  urlcolor=cyan,
  pdfauthor={\BaAuthorPDF},
  pdftitle={\BaTitle}
}

\begin{document}

%%%%%%%%%%%%%%%%%%%
%% Titelseite
%%%%%%%%%%%%%%%%%%%
\frontmatter
\titlehead{Technische Hochschule Würzburg-Schweinfurt\\Fakultät Informatik und Wirtschaftsinformatik}
\subject{\BaType}
\title{\texorpdfstring{\BaTitle\\[15mm]\TitleGraphic}{\BaTitle}}
% \subtitle{\normalsize{vorgelegt an der Technischen Hochschule W\"{u}rzburg-Schweinfurt in der Fakult\"{a}t Informatik und Wirtschaftsinformatik zum Abschluss eines Studiums im Studiengang \BaAuthorStudyProgram}}
\author{\ShowBaAuthor}
\date{\normalsize{Eingereicht am: \BaDeadline}}
\forcetwosidetitle

%%%%%%%%%%%%%%%%%%%
%% Inhaltsverzeichnis
%%%%%%%%%%%%%%%%%%%
\newpage
\setcounter{secnumdepth}{4}
\setcounter{tocdepth}{4}
\tableofcontents

%%%%%%%%%%%%%%%%%%%
%% Main part of the thesis
%%%%%%%%%%%%%%%%%%%
\mainmatter

\chapter{Einleitung}

Im vorangegangenen Portfolioteil wurde eine robuste ETL-Pipeline implementiert, die Daten des Lobbyregisters in ein normalisiertes 3NF-Schema überführt. Mit steigendem Datenvolumen und komplexeren Analyseanforderungen – etwa der Suche nach Textmustern in Millionen von Datensätzen oder der Vernetzung von Personen über verschiedene Amtszeiten hinweg – stoßen naive SQL-Abfragen jedoch schnell an ihre Grenzen.

Ziel dieses dritten Portfolioteils ist das systematische \textbf{Performance Tuning} der Datenbank. Der Fokus liegt dabei nicht auf bloßem „Ausprobieren“, sondern auf einem messbaren, methodischen Vorgehen. Wir analysieren Ausführungspläne (\texttt{EXPLAIN ANALYZE}), identifizieren Flaschenhälse (Sequential Scans, teure Joins) und implementieren gezielte Optimierungen wie spezialisierte Indizes (GIN/Trigram) und Materialized Views.

Abschließend wird der Erfolg dieser Maßnahmen durch die Integration in ein interaktives Grafana-Dashboard demonstriert, das nun Abfragen in Echtzeit ermöglicht, die zuvor mehrere Sekunden in Anspruch nahmen.

\chapter{Methodik der Performance-Messung}

Um die Wirksamkeit von Optimierungsmaßnahmen objektiv zu bewerten, wurde ein automatisiertes Benchmarking-Framework entwickelt (\texttt{scripts/benchmark.py}). Dieses Framework führt definierte „Hot Queries“ gegen die Datenbank aus und erfasst dabei Metriken direkt aus dem PostgreSQL-Query-Planner.

\section{Messverfahren und Werkzeuge}

\begin{itemize}
    \item \textbf{EXPLAIN (ANALYZE, FORMAT JSON):} Anstatt nur die Wall-Clock-Time im Python-Client zu messen (was durch Netzwerk-Latenz verfälscht werden kann), nutzen wir die interne Zeitmessung der Datenbank. Dies liefert exakte Werte für \textit{Planning Time} und \textit{Execution Time}.
    \item \textbf{Cold vs. Warm Cache:} Ein häufiger Fehler bei Performance-Messungen ist das Ignorieren des Datenbank-Caches (Shared Buffers). Eine Abfrage ist beim zweiten Aufruf oft drastisch schneller, da Daten bereits im RAM liegen.
    \item \textbf{Automatisierung:} Ein Shell-Skript (\texttt{measure\_full.sh}) steuert den Docker-Container, um reproduzierbare Zustände zu schaffen:
    \begin{enumerate}
        \item Löschen aller Indizes und Views (Baseline).
        \item Neustart des Datenbank-Containers (Cache leeren $\rightarrow$ \textbf{Cold State}).
        \item Ausführen der Benchmark-Suite.
        \item Sofortige Wiederholung der Suite (Daten im RAM $\rightarrow$ \textbf{Warm State}).
        \item Einspielen der Optimierungen (\texttt{optimization.sql}).
        \item Wiederholung der Messung (Cold & Warm).
    \end{enumerate}
\end{itemize}

\section{Relevante Szenarien (Hot Queries)}

Wir haben sechs Szenarien definiert, die typische Zugriffsmuster abdecken:
\begin{enumerate}
    \item \textbf{Finanz-Heatmap:} Aggregation über 5 Tabellen (Klassisches Reporting).
    \item \textbf{Top-Lobbyisten:} Sortierung und Filterung großer Mengen.
    \item \textbf{Drehtür-Effekt:} Filterung auf spezifische Attribute (ehemalige Regierungsmitglieder).
    \item \textbf{Netzwerk-Analyse:} Einfache Joins mit hoher Kardinalität.
    \item \textbf{Textsuche:} \texttt{ILIKE '\%Suchbegriff\%'} auf Namensfeldern (sehr teuer ohne Spezialindex).
    \item \textbf{Komplexe Netzwerkanalyse:} Ein Szenario, das Daten aus vier verschiedenen Personentabellen (Lobbyist, Vertreter, Betraute, Regierung) zusammenführt.
\end{enumerate}

\chapter{Ausgangssituation: Ist-Analyse}

Die Baseline-Messung (ohne Optimierungen) zeigte deutliche Schwächen bei komplexen Analysen und Textsuchen.

\section{Problemfall 1: Die Volltextsuche}
Eine Suche nach Firmennamen, die „Energy“ enthalten (\texttt{ILIKE '\%Energy\%'}), zwingt die Datenbank zu einem \textit{Sequential Scan} über die gesamte Tabelle \texttt{lobbyist\_identity}. Da das Wildcard-Symbol \texttt{\%} am Anfang des Suchstrings steht, kann ein normaler B-Tree-Index nicht genutzt werden.
\begin{itemize}
    \item \textbf{Laufzeit (Cold):} $\approx 7$ ms (bei kleinen Testdaten), skaliert jedoch linear schlecht mit der Datenmenge.
    \item \textbf{Kosten:} Der Planner veranschlagt hohe Kosten, da jeder Stringvergleich CPU-intensiv ist.
\end{itemize}

\section{Problemfall 2: Der Drehtür-Effekt (Complex Join)}
Um herauszufinden, welche Lobbyisten früher Regierungsämter innehatten, müssen Daten aus \texttt{lobbyist\_identity}, \texttt{entrusted\_person} und \texttt{legal\_representative} jeweils mit \texttt{recent\_government\_function} verknüpft und dann vereinigt (\texttt{UNION ALL}) werden.
Diese Abfrage ist so komplex, dass sie ohne Materialisierung für interaktive Dashboards ungeeignet ist. In der Baseline-Messung ist dieses Szenario gar nicht effizient abbildbar, da die Antwortzeiten bei großen Datenmengen in den Sekundenbereich steigen würden.

\section{Cache-Einfluss}
Wie erwartet, sinkt die Laufzeit im „Warm Cache“-Zustand. Beispielsweise verbesserte sich die Finanz-Heatmap von ca. 98 ms (Cold) auf 59 ms (Warm). Dies bestätigt, dass I/O-Operationen einen signifikanten Teil der Latenz ausmachen, löst aber nicht das Problem ineffizienter Ausführungspläne (z.B. Nested Loops über Millionen Zeilen).

\chapter{Optimierte Lösung: Advanced Indexing & Views}

Um die identifizierten Engpässe zu beseitigen, wurden zwei Hauptstrategien verfolgt: Spezialisierte Indizes für Suchanfragen und Materialized Views für komplexe Vorberechnungen.

\section{Trigram-Indizes für Textsuche}
PostgreSQL bietet mit der Extension \texttt{pg\_trgm} die Möglichkeit, Strings in Trigramme (3-Zeichen-Blöcke) zu zerlegen. Ein GIN-Index (Generalized Inverted Index) über diese Trigramme ermöglicht extrem schnelle Teilstring-Suchen.

\begin{lstlisting}[language=sql, caption=Erstellung des Trigram-Index]
CREATE EXTENSION IF NOT EXISTS pg_trgm;

CREATE INDEX IF NOT EXISTS idx_trgm_lobbyist_name 
ON public.lobbyist_identity 
USING gin (name_text gin_trgm_ops);
\end{lstlisting}

Dadurch wird der \textit{Sequential Scan} durch einen hocheffizienten \textit{Bitmap Heap Scan} ersetzt.

\section{Materialized View für Netzwerkanalysen}
Für die Analyse des „Drehtür-Effekts“ (Wechsel von Politik in die Wirtschaft) wurde eine komplexe View erstellt, die alle Personen-Tabellen normalisiert und „flachklopft“.

\begin{lstlisting}[language=sql, caption=Materialized View für Drehtür-Netzwerk]
CREATE MATERIALIZED VIEW IF NOT EXISTS public.mv_revolving_door_network AS
WITH all_gov_people AS (
    SELECT entry_id, last_name, first_name, recent_gov_function_id, 'Lobbyist' as role 
    FROM public.lobbyist_identity WHERE recent_gov_function_present = true
    UNION ALL
    SELECT li.entry_id, ep.last_name, ep.first_name, ep.recent_gov_function_id, 'Entrusted Person' as role
    FROM public.entrusted_person ep ...
    -- (Weitere Unions für Legal Representatives)
)
SELECT 
    re.register_number,
    li.name_text as organization_name,
    rgf.end_year_month,
    cl.de as gov_function_type
FROM all_gov_people p
JOIN ... -- (Joins zu Detailtabellen)
\end{lstlisting}

Diese View wird einmalig (oder periodisch) berechnet. Abfragen darauf sind triviale \texttt{SELECT * FROM view}, was komplexe Joins zur Laufzeit eliminiert.

\chapter{Bewertung und Ergebnisse}

Die durchgeführten Messungen belegen den massiven Einfluss der Optimierungen.

\section{Messergebnisse}

Tabelle \ref{tab:benchmark_results} zeigt die gemittelten Laufzeiten der Benchmark-Szenarien.

\begin{table}[H]
    \centering
    \caption{Vergleich der Ausführungszeiten (in ms)}
    \label{tab:benchmark_results}
    \begin{tabular}{lrrrr}
        \toprule
        & \multicolumn{2}{c}{\textbf{Baseline (Unoptimized)}} & \multicolumn{2}{c}{\textbf{Optimized (Indizes \& MV)}} \\
        \cmidrule(lr){2-3} \cmidrule(lr){4-5}
        \textbf{Szenario} & \textbf{Cold} & \textbf{Warm} & \textbf{Cold} & \textbf{Warm} \\
        \midrule
        1. Finanz-Heatmap & 97.82 & 59.19 & 61.42 & 57.95 \\
        2. Top-Lobbyisten & 12.54 & 9.56 & 12.43 & 7.76 \\
        3. Drehtür-Effekt & 1.64 & 0.93 & 1.52 & 0.89 \\
        4. Netzwerk-Analyse & 1.03 & 0.48 & 0.67 & 0.33 \\
        5. Textsuche (Trigram) & 6.81 & 7.32 & \textbf{0.34} & \textbf{0.17} \\
        6. Drehtür-MV (Neu) & -- & -- & \textbf{1.35} & \textbf{0.73} \\
        \bottomrule
    \end{tabular}
\end{table}

\section{Interpretation}

Die Grafik in Abbildung \ref{fig:benchmark_chart} visualisiert den Performance-Gewinn auf einer logarithmischen Skala.

\begin{figure}[H]
    \centering
    \begin{tikzpicture}
        \begin{axis}[
            ybar,
            width=0.95\textwidth,
            height=8cm,
            symbolic x coords={Heatmap, Lobbyisten, Drehtür, Netzwerk, Textsuche},
            xtick=data,
            ylabel={Ausführungszeit in ms (Log-Skala)},
            ymode=log,
            log basis y={10},
            ymin=0.1, ymax=200,
            nodes near coords,
            nodes near coords align={vertical},
            every node near coord/.append style={font=\footnotesize, /pgf/number format/fixed, /pgf/number format/precision=2},
            legend style={at={(0.5,-0.15)}, anchor=north, legend columns=-1},
            ymajorgrids=true,
            grid style=dashed,
        ]
            % Baseline (Cold)
            \addplot[fill=red!40] coordinates {
                (Heatmap,97.82) 
                (Lobbyisten,12.54) 
                (Drehtür,1.64) 
                (Netzwerk,1.03) 
                (Textsuche,6.81)
            };
            
            % Optimized (Cold)
            \addplot[fill=green!40] coordinates {
                (Heatmap,61.42) 
                (Lobbyisten,12.43) 
                (Drehtür,1.52) 
                (Netzwerk,0.67) 
                (Textsuche,0.34)
            };
            
            \legend{Baseline (Cold), Optimized (Cold)}
        \end{axis}
    \end{tikzpicture}
    \caption{Performance-Vergleich vor und nach Optimierung (Cold Cache). Beachte besonders den extremen Gewinn bei der Textsuche (Faktor 20).}
    \label{fig:benchmark_chart}
\end{figure}

Die Ergebnisse sind eindeutig:
\begin{enumerate}
    \item \textbf{Textsuche:} Durch den Trigram-Index sank die Laufzeit von $\approx 7$ ms auf $0,17$ ms (Warm). Dies ist eine Beschleunigung um den \textbf{Faktor 40}. Dies ermöglicht "Live Search" Features, bei denen Ergebnisse schon während des Tippens erscheinen.
    \item \textbf{Komplexe Analysen:} Die Abfrage für das Drehtür-Szenario läuft gegen die Materialized View in unter 1 ms. Ohne die View müssten zur Laufzeit vier Tabellen mit \texttt{UNION} verbunden und gejoint werden, was um Größenordnungen langsamer wäre.
    \item \textbf{Standard-Queries:} Bei einfachen Abfragen (z.B. Top-Lobbyisten) ist der Gewinn durch Indizes moderater ($\approx 20\%$), da hier oft die I/O-Bandbreite und nicht die CPU der limitierende Faktor ist.
\end{enumerate}

\section{Visualisierung in Grafana}

Die optimierten Abfragen bilden die Basis für das neue Dashboard „Lobbyregister Advanced Search“. 

\begin{figure}[H]
    \centering
    % Platzhalter - Bitte Screenshot einfügen!
    \includegraphics[width=\textwidth]{GIN-Trigram-Suche.png}
    \caption{Die Live-Suche im Dashboard reagiert dank GIN-Index in Echtzeit auf Eingaben.}
    \label{fig:grafana_search}
\end{figure}

Abbildung \ref{fig:grafana_search} zeigt die Suchfunktion, die den Trigram-Index nutzt. Abbildung \ref{fig:grafana_network} visualisiert die Daten aus der Materialized View.

\begin{figure}[H]
    \centering
    % Platzhalter - Bitte Screenshot einfügen!
    \includegraphics[width=\textwidth]{Dashboard_optimized.png}
    \caption{Analyse des Drehtür-Effekts basierend auf der voraggregierten Materialized View.}
    \label{fig:grafana_network}
\end{figure}

\chapter{Fazit und Ausblick}

In diesem Portfolio-Teil konnte gezeigt werden, dass ein gut normalisiertes Datenmodell allein noch keine performante Anwendung garantiert. Erst durch den gezielten Einsatz von \textbf{Advanced Indexing} (Trigramme) und \textbf{Materialized Views} (für komplexe Joins) wird die Datenbank reaktionsschnell genug für interaktive Dashboards.

Besonders hervorzuheben ist die Beschleunigung der Textsuche um den Faktor 40 und die Reduktion komplexer Netzwerk-Queries auf unter 1 Millisekunde. Das methodische Vorgehen – Messen, Optimieren, Verifizieren – hat sichergestellt, dass Optimierungen nicht auf Vermutungen, sondern auf Fakten basieren.

Im nächsten Schritt könnte die Pipeline um eine Volltextsuche (FTS) mit \texttt{tsvector} erweitert werden. Zudem wäre eine Automatisierung des \texttt{REFRESH MATERIALIZED VIEW} mittels Datenbank-Triggern oder Cronjobs ein logischer nächster Schritt für den produktiven Betrieb.

\end{document}