\documentclass[12pt,oneside,a4paper,parskip=half]{scrbook}

% Sprachanpassung und Grundkonfiguration
\usepackage[utf8]{inputenc}
\usepackage[T1]{fontenc}
\usepackage[ngerman]{babel}
\usepackage{lmodern}
\usepackage{microtype}
\usepackage{csquotes}
\usepackage{hyphenat}
\usepackage{newunicodechar}
\newunicodechar{ }{\,}

% Seitenlayout
\usepackage[a4paper,left=20mm,right=20mm,top=20mm,bottom=25mm]{geometry}
\usepackage{setspace}
\onehalfspacing
\sloppy

% Diagramme
\usepackage{pgfplots}
\pgfplotsset{compat=1.17}

% Mathematik & Symbole
\usepackage{amsmath,amsfonts,amssymb}

% Tabellen & Grafiken
\usepackage{graphicx}
\usepackage{float}
\usepackage{booktabs}
\usepackage{longtable}
\usepackage{tabularx}
\usepackage{pdflscape}
\usepackage{placeins}
\graphicspath{{./}{./figures/}}

% Aufzählungen
\usepackage{enumitem}

% Farben (nur noch für Links oder Basics)
\usepackage{xcolor}
\usepackage{listings}
\lstdefinelanguage{SQL}{
  sensitive=false,
  morekeywords={
    SELECT,FROM,WHERE,JOIN,LEFT,RIGHT,INNER,OUTER,FULL,ON,GROUP,BY,ORDER,ASC,DESC,
    COUNT,AS,CASE,WHEN,THEN,ELSE,END,DISTINCT,UNION,ALL,LIMIT,OFFSET,
    AND,OR,NOT,NULL,IS,COALESCE,ROUND,AVG,SUM,MIN,MAX,OVER,PARTITION,BY
  },
  morecomment=[l]{--},
  morecomment=[s]{/*}{*/},
  morestring=[b]'
}
\lstdefinestyle{sql}{
  language=SQL,
  basicstyle=\ttfamily\small,
  keywordstyle=\color{blue!60!black}\bfseries,
  commentstyle=\color{gray!70},
  stringstyle=\color{orange!70!black},
  numbers=left,
  numberstyle=\tiny\color{gray!70},
  numbersep=6pt,
  columns=fullflexible,
  keepspaces=true,
  frame=single,
  rulecolor=\color{black!15},
  backgroundcolor=\color{black!2},
  breaklines=true,
  showstringspaces=false,
  inputencoding=utf8,
  extendedchars=true,
  literate=
    {ä}{{\"a}}1 {ö}{{\"o}}1 {ü}{{\"u}}1
    {Ä}{{\"A}}1 {Ö}{{\"O}}1 {Ü}{{\"U}}1
    {ß}{{\ss}}1
    {€}{{\texteuro}}1
    {Ø}{{\O}}1
}

%%%%%%%%%%%%%%%%%%%
%% definitions
%%%%%%%%%%%%%%%%%%%
\def\BaAuthor{Noah Raupold (5022097),\\ David Gläsle (5022114)}
\def\BaAuthorPDF{Noah Raupold (5022097), David Gläsle (5022114)}
\def\BaAuthorStudyProgram{Informatik}
\def\BaType{ADT Portfolio Teil 4}
\def\BaTitle{Visualisierung des Lobbyregisters}
\def\BaDeadline{\today}

\def\iswithfullname{1}
\ifdefined\iswithfullname
\def\ShowBaAuthor{\BaAuthor}
\else
\def\ShowBaAuthor{N.~N.}
\fi

\newcommand{\TitleGraphic}{
\IfFileExists{logo.png}{
\includegraphics[width=0.5\textwidth]{logo.png}%
}{
\fbox{\parbox[c][3cm][c]{0.5\textwidth}{Logo.png fehlt}}%
}% 
}

\newcommand*{\forcetwosidetitle}{
\begingroup
\cleardoubleoddpage
\KOMAoptions{titlepage=true}%
\csname @twosidetrue\endcsname
\maketitle
\endgroup
}

\newcommand{\TOCbreak}{\\
}

% Bibliografie (Biber)
\usepackage[backend=biber,style=numeric]{biblatex}
\IfFileExists{literatur.bib}{\addbibresource{literatur.bib}}{}

% Hyperlinks
\usepackage{hyperref}
\hypersetup{
colorlinks=true,
linkcolor=black,
filecolor=magenta,
urlcolor=cyan,
pdfauthor={\BaAuthorPDF},
pdftitle={\BaTitle}
}

\begin{document}

%%%%%%%%%%%%%%%%%%%
%% Titelseite
%%%%%%%%%%%%%%%%%%%
\frontmatter
\titlehead{Technische Hochschule Würzburg-Schweinfurt\\Fakultät Informatik und Wirtschaftsinformatik}
\subject{\BaType}
\title{\texorpdfstring{\BaTitle\\[15mm]\TitleGraphic}{\BaTitle}}
\author{\ShowBaAuthor}
\date{\normalsize{Eingereicht am: \BaDeadline}}
\forcetwosidetitle

%%%%%%%%%%%%%%%%%%%
%% Inhaltsverzeichnis
%%%%%%%%%%%%%%%%%%%
\newpage
\setcounter{secnumdepth}{4}
\setcounter{tocdepth}{4}
\tableofcontents

%%%%%%%%%%%%%%%%%%%
%% Main part of the thesis
%%%%%%%%%%%%%%%%%%%
\mainmatter
\chapter{Einleitung: Vom Datengrab zum Dashboard}

Nach dem Aufbau der ETL-Strecke und der optimierten Datenbankarchitektur (Indexing, Materialized Views) richtet sich der Abschluss dieses Portfolios auf die Visualisierungsschicht. Tausende Tabellenzeilen sind für Entscheidungen wertlos, solange sie nicht verdichtet, in Kontext gesetzt und visuell lesbar gemacht werden.

Ziel dieses Teils ist es, die technische Tiefe der PostgreSQL-Datenbank (Window Functions, CTEs\footnote{Common Table Expressions.}, rekursive Abfragen) in prägnante Dashboards zu übersetzen. Grafana passt hier, weil es sich reibungslos in den bestehenden Docker-Stack einfügt und SQL nicht hinter einer Oberfläche versteckt. Sichtbar werden sollen versteckte Netzwerke („Wer kennt wen?“), finanzielle Ausreißer und thematische Schwerpunkte im deutschen Lobbyregister.

Die Visualisierung steht damit am Ende einer Kette von Entscheidungen, die in den vorangegangenen Portfolios getroffen wurden: Datenmodellierung nach strengem Normalisierungsgrad, robuste ETL-Pipeline mit \texttt{asyncio} und gezielte Performance-Tuning-Maßnahmen. Erst auf dieser Basis können Dashboards belastbare Aussagen treffen, ohne sich auf Voraggregationen oder manuelle Datenaufbereitung verlassen zu müssen.

Adressaten der Visualisierungen sind unterschiedliche Gruppen: politische Entscheidungsträger, die einen schnellen Überblick benötigen; Analystinnen, die explorative Drill-downs durchführen; und Öffentlichkeit beziehungsweise Medien, die Transparenz über Verflechtungen erwarten. Alle drei profitieren von reproduzierbaren Panels, die unmittelbar aus der Datenbank gespeist werden und so eine konsistente „Single Source of Truth“ wahren.

\chapter{Visualisierungsstrategie mit Grafana}

Grafana bietet – anders als klassische BI-Tools wie Tableau oder PowerBI – einen direkten, code-basierten Zugriff auf die Datenbank. So bleibt die volle SQL-Funktionalität erhalten, statt auf generische Drag-and-Drop-Aggregationen reduziert zu werden. Der Grafana-Container ist in den bestehenden Docker-Stack eingebunden und greift über eine dedizierte PostgreSQL-Datasource auf die Materialized Views und Indizes zu, die in den vorherigen Phasen erarbeitet wurden.

Im Fokus stehen reproduzierbare Panels mit klar definierten SQL-Queries, die ohne proprietäre Berechnungen auskommen. Durch diesen Ansatz bleibt das gesamte Datenmodell transparent; jede Kennzahl lässt sich auf eine Query zurückführen, die versioniert und nachvollziehbar dokumentiert ist.

\section{Architektur der Dashboards}
Unsere Visualisierungsstrategie folgt dem „Schneidenbohrer-Prinzip“ (Drill-Down):

\begin{enumerate}
    \item \textbf{High-Level Overview:} 
    Der Einstieg erfolgt über ein globales Dashboard, das die wichtigsten KPIs aggregiert. Entscheider sehen dort sofort die Gesamtzahl der aktiven Lobbyisten, das kumulierte Finanzvolumen sowie die personelle Schlagkraft (FTE\footnote{Full Time Equivalent.}).
    
    \begin{figure}[H]
        \centering
        \includegraphics[width=\textwidth]{figures/dashboard_overview.png}
        \caption{Das Overview-Dashboard: Zentrale KPIs, FTE-Analyse und Themenfelder auf einen Blick.}
        \label{fig:dash_overview}
    \end{figure}

    \item \textbf{Analytical Deep-Dive (Geo \& Organisation):} 
    Für detaillierte Analysen bieten spezialisierte Dashboards tiefe Einblicke. Das Geo-Dashboard visualisiert die Verteilung der Akteure auf einer Karte.
    
    \begin{figure}[H]
        \centering
        \includegraphics[width=\textwidth]{figures/dashboard_geo.png}
        \caption{Geografische Verteilung der Lobby-Akteure (Inland \& Ausland).}
        \label{fig:dash_geo}
    \end{figure}

    Besonders wertvoll ist der Drill-Down auf Städte-Ebene. Im Beispiel \textbf{Würzburg} (Abb. \ref{fig:dash_city}) lässt sich gezielt untersuchen, welche Akteure in einer spezifischen Region ansässig sind und wie hoch deren lokales Finanzvolumen ist.
    
    \begin{figure}[H]
        \centering
        \includegraphics[width=\textwidth]{figures/dashboard_city.png}
        \caption{Detailansicht "City": Filterung am Beispiel der Stadt Würzburg.}
        \label{fig:dash_city}
    \end{figure}

    Ebenso erlaubt der Organization-Profiler die Durchleuchtung einzelner Akteure. Am Beispiel der \textbf{Fraunhofer-Gesellschaft zur Förderung der angewandten Forschung e. V.} (Abb. \ref{fig:dash_org}) werden Mitarbeiterzahlen, Finanzhistorie, Mitgliedschaften und spezifische Gesetzesvorhaben transparent gemacht.

    \begin{figure}[H]
        \centering
        \includegraphics[width=\textwidth]{figures/dashboard_organization.png}
        \caption{Organization-Profiler: 360°-Sicht am Beispiel der Fraunhofer-Gesellschaft.}
        \label{fig:dash_org}
    \end{figure}

    \item \textbf{Forensische Detailansicht \& Netzwerke:} 
    Abschließend ermöglichen Spezial-Dashboards die Prüfung auf Compliance-Verstöße und Netzwerke. Das Transparenz-Dashboard listet Akteure, die Finanzangaben verweigern oder Jahresberichte schuldig bleiben.
    
    \begin{figure}[H]
        \centering
        \includegraphics[width=\textwidth]{figures/dashboard_compliance.png}
        \caption{Transparenz-Dashboard: Identifikation von Akteuren mit verweigerten Angaben.}
        \label{fig:dash_transparenz}
    \end{figure}
    
    Das Advanced-Analytics Dashboard (Abb. \ref{fig:dash_advanced}) visualisiert komplexe Zusammenhänge wie legislative Fußabdrücke und Netzwerke.
    
    \begin{figure}[H]
        \centering
        \includegraphics[width=\textwidth]{figures/dashboard_advanced.png}
        \caption{Advanced Analytics: Netzwerkanalysen und legislative Zusammenhänge.}
        \label{fig:dash_advanced}
    \end{figure}
\end{enumerate}

\section{Technische Herausforderungen}
Knackpunkte waren die Visualisierung von Netzwerken und Zeitreihen, weil das Lobbyregister häufig nur Monats- und Jahresangaben liefert. Durch SQL-Casts (\texttt{TO\_DATE}) und Fallback-Logiken haben wir dennoch stabile Zeitachsen für Grafana aufgebaut.

\chapter{Implementierung komplexer Analysen}

Im Mittelpunkt stehen nicht die Charts, sondern die SQL-Abfragen, die sie speisen. Grafana fungiert hier als Visualisierungsschicht, bleibt aber eng an der Datenbank: Alle Metriken entstehen aus versionierten Queries, die direkt auf den Materialized Views und Indizes aufsetzen. Damit sind die Analysen reproduzierbar und können sowohl im Dashboard als auch in isolierten SQL-Umgebungen ausgeführt werden.

Die folgenden Beispiele zeigen drei Abfrageklassen, die über einfache Aggregationen hinausgehen: mehrstufige Joins zur Rekonstruktion von Personen- und Organisationsnetzwerken, Window Functions für Rankings sowie Text- und Datumsnormalisierung, um unvollständige Eingaben auszuwerten. Sie illustrieren, wie wir Transparenz, Performanz und fachliche Aussagekraft miteinander verbinden.

\section{Network of Influence: Die unsichtbaren Verbindungen}
Transparenz scheitert oft an indirekten Ketten: Ein Unternehmen (Client) beauftragt eine Agentur, die wiederum einen ehemaligen Minister beschäftigt. Sichtbar wird diese Spur mit einem 5-Wege-Join.

\begin{figure}[H]
\caption{SQL-Query für das 'Network of Influence' Panel}
\begin{lstlisting}[style=sql]
SELECT
    cco.name AS "Auftraggeber (Client)",
    li.name_text AS "Beauftragte Agentur",
    -- Fallback: Name aus Personentabelle oder Organisationstabelle
    COALESCE(ep.last_name || ', ' || ep.first_name, li2.name_text) 
      AS "Ex-Politiker im Team",
    cl.de AS "Ehemalige Funktion"
FROM contract_client_org cco
-- Der Weg des Geldes: Client -> Vertrag -> Agentur
JOIN contract_clients cc ON cco.clients_id = cc.id
JOIN contract_item c_item ON cc.contract_item_id = c_item.id
JOIN contracts c ON c_item.parent_id = c.id
JOIN register_entry re ON c.entry_id = re.id
JOIN lobbyist_identity li ON re.id = li.entry_id
-- Der personelle Link: Agentur -> Ex-Politiker
LEFT JOIN entrusted_person ep ON li.id = ep.identity_id
LEFT JOIN legal_representative lr ON li.id = lr.identity_id
JOIN recent_government_function rgf 
    ON (ep.recent_gov_function_id = rgf.id 
        OR lr.recent_gov_function_id = rgf.id)
LEFT JOIN code_label cl ON rgf.type_label_id = cl.id
WHERE rgf.id IS NOT NULL;
\end{lstlisting}
\end{figure}

Ein konkreter Lauf der Abfrage zeigte beispielsweise:

\begin{itemize}
    \item \textbf{Auftraggeber:} Arena2036
    \item \textbf{Beauftragte Agentur:} Strategische Agentur für Innovation in Europa (SAI Europe)
    \item \textbf{Verbindung:} Bundestag (via ehemaligem Funktionsträger)
\end{itemize}

Dies zeigt exemplarisch, wie Forschungscampus-Projekte (Arena2036) über spezialisierte Agenturen Zugang zu politischen Entscheidern suchen.

\section{Finanzielle Rankings: Window Functions im Einsatz}
Um die finanzkräftigsten Akteure zu identifizieren, setzen wir SQL Window Functions ein. \texttt{RANK()} erzeugt eine dynamische Bestenliste direkt in der Datenbank; ein Umweg über die Applikation entfällt.

\begin{figure}[H]
\caption{Berechnung des Budget-Rankings mittels RANK()}
\begin{lstlisting}[style=sql]
SELECT
    name,
    expenses,
    rank
FROM (
    SELECT
        COALESCE(li.name_text, li.company_name) AS name,
        fe.expenses_to_eur AS expenses,
        -- Ranking basierend auf Ausgaben
        RANK() OVER (
            ORDER BY fe.expenses_to_eur DESC NULLS LAST
        ) AS rank
    FROM financial_expenses fe
    JOIN lobbyist_identity li ON fe.entry_id = li.entry_id
) AS ranked_stats
WHERE rank <= 3;
\end{lstlisting}
\end{figure}

Das Ergebnis der Analyse zeigt die "Heavy Hitter" des deutschen Lobbyismus (Stand: aktueller Datenabzug):

\begin{table}[H]
\centering
\begin{tabular}{clr}
\toprule
\textbf{Rang} & \textbf{Organisation} & \textbf{Budget (EUR)} \\
\midrule
1 & Gesamtverb. d. Dt. Versicherungswirtschaft & 15.300.000 \\
2 & Verbraucherzentrale Bundesverband e.V. & 12.740.000 \\
3 & Ramboll Management Consulting & 12.620.000 \\
\bottomrule
\end{tabular}
\caption{Top 3 Lobbyisten nach gemeldeten Finanzaufwendungen.}
\end{table}

\chapter{Data Stories: Erkenntnisse aus den Daten}

Neben der Implementierung haben wir die Datenbank explorativ („Data Mining“) genutzt, um Muster im deutschen Lobbyismus freizulegen. Die folgenden Abschnitte fassen ausgewählte Befunde zusammen.

\section{Die „Drehtür“-Ministerien}
Wir haben analysiert, welche Bundesministerien oder Behörden am häufigsten in den Lebensläufen von Lobbyisten auftauchen (Tabelle \ref{tab:ministries}).

\begin{table}[H]
\centering
\begin{tabular}{llc}
\toprule
\textbf{Kürzel} & \textbf{Ministerium / Behörde} & \textbf{Anzahl Treffer} \\
\midrule
BMVg & Bundesministerium der Verteidigung & 6 \\
BMU & Bundesministerium für Umwelt & 3 \\
BMAS & Bundesministerium für Arbeit & 2 \\
BMJ & Bundesministerium der Justiz & 2 \\
BMBFSFJ & Bundesministerium für Familie & 1 \\
\bottomrule
\end{tabular}
\caption{Top 5 Herkunftsorte von Lobbyisten mit Regierungshintergrund. Das Verteidigungsministerium führt die Liste an.}
\label{tab:ministries}
\end{table}

Es ist auffällig, dass das Verteidigungsressort dominiert, gefolgt von Umwelt- und Arbeitsthemen.

\section{Die teuersten Themenfelder}
Welche politischen Themen sind den Lobbyisten das meiste Geld wert? Wir berechnen das Durchschnittsbudget aller Organisationen, die ein bestimmtes Interessenfeld angeben. Das gesamte Finanzvolumen im Register liegt bei \textbf{909 Millionen Euro}.

\begin{figure}[H]
\centering
\begin{tikzpicture}
\begin{axis}[
    xbar,
    xlabel={Durchschnittliches Budget (EUR)},
    symbolic y coords={{Außenpolitik (EU)}, Strafrecht, Bevölkerungsschutz, Cybersicherheit, Digitalisierung},
    ytick=data,
    nodes near coords,
    nodes near coords align={horizontal},
    width=0.8\textwidth,
    height=8cm,
    xmin=0,
    xmax=800,
    xtick distance=100,
    enlarge y limits=0.2,
    scaled x ticks=false,
    xticklabel={\pgfmathprintnumber[fixed,precision=0]{\tick}k\,€},
    nodes near coords={\pgfmathprintnumber[fixed,precision=0]{\pgfplotspointmeta}k\,€}
]
\addplot coordinates {
    (432,{Außenpolitik (EU)})
    (449,Strafrecht)
    (518,Bevölkerungsschutz)
    (519,Cybersicherheit)
    (642,Digitalisierung)
};
\end{axis}
\end{tikzpicture}
\caption{Top 5 Themenfelder mit dem höchsten Durchschnittsbudget pro Lobby-Akteur.}
\label{fig:top_topics}
\end{figure}

\begin{itemize}
    \item \textbf{Platz 1: Digitalisierung (Ø 641.667 €)} – Das Megathema unserer Zeit zieht massives Kapital an.
    \item \textbf{Platz 2: Cybersicherheit (Ø 519.129 €)} – Angesichts wachsender Bedrohungen ist die digitale Verteidigung das teuerste Pflaster des Lobbyismus (495 Akteure).
    \item \textbf{Platz 3: Bevölkerungsschutz und Katastrophenhilfe (Ø 517.532 €)} – Ein Bereich, der oft staatliche Großaufträge impliziert und entsprechend finanzstarke Akteure anzieht (246 Akteure).
\end{itemize}
Klassische Industriethemen (Auto, Chemie) sind in Summe groß, werden im \textit{Durchschnitt} aber durch viele kleine Zulieferer verwässert. Die oben genannten Felder werden dagegen fast ausschließlich von „Big Playern“ bespielt.

Die Daten zeigen, dass diese teuren Themenfelder von finanzstarken Verbänden dominiert werden. Oft streuen sie ihre hohen Budgets breit:
\begin{itemize}
    \item \textbf{Gesamtverband der Deutschen Versicherungswirtschaft e.V.} (15,3 Mio. € Jahresbudget) treibt den Durchschnitt im Bereich Strafrecht nach oben, verteilt seine Ressourcen jedoch auf insgesamt \textbf{37 Themenfelder}.
    \item \textbf{Verbraucherzentrale Bundesverband e.V.} (12,7 Mio. €) ist federführend im Bereich Cybersicherheit, bespielt aber gleichzeitig \textbf{46 weitere Themen}.
    \item \textbf{Verband der Automobilindustrie e.V.} (9,9 Mio. €) ist in allen drei Top-Kategorien aktiv, streut sein Budget jedoch am breitesten über \textbf{60 verschiedene Interessenfelder}.
\end{itemize}

\section{Der Gesetzgebungs-Fußabdruck}
Das Lobbyregister erfasst nicht nur Interessenfelder, sondern auch konkrete Gesetzesvorhaben (\textit{Regulatory Projects}). Damit lässt sich messen, welche Gesetze die meisten Lobbyisten mobilisieren.

\begin{figure}[H]
\caption{Identifikation der umkämpftesten Gesetzesvorhaben}
\begin{lstlisting}[style=sql]
SELECT 
    rpi.title as "Gesetzesvorhaben", 
    COUNT(DISTINCT re.id) as "Lobbyisten-Anzahl"
FROM register_entry re
JOIN regulatory_projects rp ON re.id = rp.entry_id
JOIN regulatory_project_item rpi ON rp.id = rpi.parent_id
-- Filterung nach Relevanz (z.B. Digitalisierung)
JOIN activities_interests ai ON re.id = ai.entry_id
JOIN field_of_interest foi ON ai.id = foi.activities_id
JOIN code_label cl ON foi.label_id = cl.id
WHERE cl.de = 'Digitalisierung'
GROUP BY rpi.title
ORDER BY 2 DESC
LIMIT 5;
\end{lstlisting}
\end{figure}

So werden politische Konfliktlinien sichtbar, bevor sie breit diskutiert werden. Ein Lauf der Abfrage für den Sektor „Digitalisierung“ zeigte überraschend: Nicht ein IT-Sicherheitsgesetz, sondern die \textbf{„Praxisorientierte Anpassung des Energieeffizienzgesetzes“} (10 betroffene Lobbyisten) führte. Das unterstreicht die Verzahnung von Digitalisierung und Energiepolitik. Auch die \textbf{EU-Plastikabgabe} und die \textbf{Digitalisierbarkeit steuerlicher Prozesse} (je 9 Lobbyisten) stehen weit oben.

\section{Effizienz-Analyse: Wer hat den größten Hebel?}
Die Metrik „Ausgaben pro Mitarbeiter“ (Spending per Head) brachte unerwartete Spitzenreiter hervor.
Großkonzerne wie \textit{Deutsche Bank} oder \textit{Allianz} haben erwartbar hohe Budgets, doch spezialisierte Kammern und Tech-Konzerne führen das Effizienz-Ranking. Die \textbf{German American Chamber of Commerce} (147.000 € pro Kopf) und \textbf{Meta Platforms} (129.000 € pro Kopf) investieren am meisten pro registriertem Lobbyisten – ein Hinweis auf wenige, aber hochspezialisierte Experten mit großen Budgets.

Zum Vergleich: Etablierte Großkonzerne agieren oft mit anderen Relationen. Die \textbf{Deutsche Bank AG} meldet beispielsweise ein Budget von 1,84 Mio. € bei nur 1,45 angegebenen Vollzeitäquivalenten (FTE), was auf den Einsatz hochbezahlter externer Berater oder weniger, aber extrem teurer Spitzenlobbyisten hindeutet. Die \textbf{Allianz SE} operiert mit 980.000 € Budget bei 3,73 FTE etwas „konservativer“.

\section{Lobbyregister Fun Facts}
Zum Abschluss noch einige kuriose Statistiken, die wir direkt aus der Datenbank extrahiert haben:

\begin{itemize}
    \item \textbf{Der Spendenkönig:} Bill Gates (Privatperson) führt die Liste der Spender mit 97,9 Mio. € unangefochten an.
    \item \textbf{Die Lobby-Hauptstädte:} Außerhalb von Berlin sitzen die meisten Lobbyisten in \textbf{München (383)}, gefolgt von Hamburg (318) und der ehemaligen Hauptstadt Bonn (230).
    \item \textbf{Die „Vielseitigsten“:} Das „Aktionsbündnis Für die Würde unserer Städte“ vertritt stolze 62 verschiedene Auftraggeber (Clients) – ein Rekordwert für Bündelung von Interessen.
    \item \textbf{Namensvettern:} Wer in Deutschland Lobbyist werden will, hat mit dem Vornamen \textbf{Andreas} (8 Treffer) statistisch die besten Chancen, dicht gefolgt von Peter und Michael.
    \item \textbf{Bestvernetzt:} Die \textit{Deutsch-Taiwanische Gesellschaft e.V.} pflegt mit 12 direkten Verbindungen zu ehemaligen Regierungsmitgliedern das dichteste Kontaktnetzwerk im Datensatz.
\end{itemize}

\chapter{Geografische Deep-Dives}

Neben der inhaltlichen Dimension ("Wer spricht worüber?") ermöglicht das Datenmodell präzise geografische Analysen. Wir wollten wissen, wo die Lobbyisten tatsächlich sitzen – und ob "Berlin" wirklich der einzige Ort ist, an dem Entscheidungen beeinflusst werden.

\section{Globale Akteure: Lobbying aus dem Ausland}
Das Lobbyregistergesetz gilt nicht nur für deutsche Unternehmen. Viele internationale Akteure versuchen, Einfluss auf die Bundespolitik zu nehmen. Unsere "International-Query" filtert alle Einträge mit Hauptsitz außerhalb Deutschlands.

\begin{figure}[H]
\caption{SQL-Query zur Analyse internationaler Herkunftsländer}
\begin{lstlisting}[style=sql]
SELECT 
  cl.code as "Land", 
  COUNT(DISTINCT re.id) as "Anzahl Akteure"
FROM register_entry re
JOIN lobbyist_identity li ON re.id = li.entry_id
JOIN address a ON li.address_id = a.id
JOIN country_label cl ON a.country_id = cl.id
WHERE cl.code != 'DE' -- Ausschluss von Deutschland
GROUP BY cl.code
ORDER BY 2 DESC
LIMIT 10;
\end{lstlisting}
\end{figure}

Das Ergebnis: Neben erwartbaren Nachbarn (Belgien/Brüssel, Österreich) tauchen auch USA und Großbritannien weit oben auf. Lobbyismus in Berlin ist damit klar international. Spitzenreiter sind \textbf{Belgien (49 Akteure)} und die \textbf{USA (47)}, gefolgt von den Niederlanden (43), Großbritannien (41) und der Schweiz (38).

\begin{figure}[H]
    \centering
    \IfFileExists{figures/worldmap.png}{\includegraphics[width=0.9\textwidth]{figures/worldmap.png}}{\fbox{Screenshot 'figures/worldmap.png' fehlt}}
    \caption{Geomap-Visualisierung der Herkunftsländer internationaler Lobbyisten (außer Deutschland).}
    \label{fig:worldmap}
\end{figure}

\begin{figure}[H]
\centering
\begin{tikzpicture}
\begin{axis}[
    xbar,
    xlabel={Anzahl Akteure},
    symbolic y coords={Schweiz, Großbritannien, Niederlande, USA, Belgien},
    ytick=data,
    nodes near coords,
    nodes near coords align={horizontal},
    width=0.8\textwidth,
    height=6cm,
    xmin=0,
    xmax=55
]
\addplot coordinates {
    (38,Schweiz)
    (41,Großbritannien)
    (43,Niederlande)
    (47,USA)
    (49,Belgien)
};
\end{axis}
\end{tikzpicture}
\caption{Top 5 Herkunftsländer internationaler Lobbyisten (ohne Deutschland).}
\label{fig:international}
\end{figure}

\section{Die PLZ-Heatmap}
Um eine geografische Verteilung innerhalb Deutschlands zu erhalten, ohne auf komplexe GIS-Systeme\footnote{Geographic Information Systems.} zurückgreifen zu müssen, haben wir eine Analyse auf Basis der ersten Ziffer der Postleitzahl durchgeführt. Dies liefert eine robuste „Heatmap“ der Lobby-Dichte.

\begin{figure}[H]
\caption{Verteilung nach Postleitzahl-Regionen}
\begin{lstlisting}[style=sql]
SELECT 
  SUBSTRING(a.zip_code, 1, 1) || 'xxxx' as "PLZ-Region",
  COUNT(DISTINCT re.id) as "Anzahl Organisationen"
FROM register_entry re
JOIN lobbyist_identity li ON re.id = li.entry_id
JOIN address a ON li.address_id = a.id
JOIN country_label cl ON a.country_id = cl.id
WHERE cl.code = 'DE' AND a.zip_code IS NOT NULL
GROUP BY SUBSTRING(a.zip_code, 1, 1)
ORDER BY 1 ASC;
\end{lstlisting}
\end{figure}

Die Ergebnisse zeigen erwartungsgemäß eine massive Konzentration im Bereich \textbf{„1xxxx“} (Berlin/Brandenburg) mit \textbf{2.072 Organisationen}.
Spannend sind die Cluster außerhalb der Hauptstadt:
\begin{itemize}
    \item \textbf{Region 5xxxx (Köln/Bonn):} Mit 721 Organisationen immer noch ein Schwergewicht – ein Erbe der Bonner Republik.
    \item \textbf{Region 8xxxx (München/Bayern):} 687 Akteure, was die wirtschaftliche Stärke des Südens widerspiegelt.
    \item \textbf{Region 6xxxx (Frankfurt/Rhein-Main):} 615 Akteure, getrieben durch den Finanzsektor.
\end{itemize}
Im Gegensatz dazu ist der Osten (PLZ 0xxxx) mit nur 221 Einträgen (außerhalb Berlins) lobbyistisch kaum repräsentiert.

\begin{figure}[H]
    \centering
    \IfFileExists{figures/plz_heatmap.png}{\includegraphics[width=0.8\textwidth]{figures/plz_heatmap.png}}{\fbox{Screenshot 'figures/plz\_heatmap.png' fehlt}}
    \caption{Verteilung der Lobby-Akteure nach Postleitzahl-Regionen (Grafana).}
    \label{fig:grafana_plz}
\end{figure}

\section{München vs. Berlin: Wo sitzt das Geld?}
Ein besonders spannendes "Data Storytelling"-Element ist der Vergleich der durchschnittlichen Finanzausstattung pro Stadt. Berlin hat die meisten Büros, doch das "große Geld" sitzt oft in den Konzernzentralen.

Wir haben dazu die Tabelle \texttt{financial\_expenses} mit der Adress-Tabelle \texttt{address} verknüpft und nach Städten gruppiert.

\begin{figure}[H]
\caption{Vergleich der durchschnittlichen Budgets nach Städten}
\begin{lstlisting}[style=sql]
SELECT 
  a.city as "Stadt",
  ROUND(AVG(fe.expenses_to_eur), 0) as "Avg Budget (EUR)"
FROM register_entry re
JOIN lobbyist_identity li ON re.id = li.entry_id
JOIN address a ON li.address_id = a.id
JOIN financial_expenses fe ON re.id = fe.entry_id
WHERE a.city IN ('Berlin', 'München', 'Hamburg', 'Frankfurt am Main')
GROUP BY a.city
ORDER BY 2 DESC;
\end{lstlisting}
\end{figure}

Das Ergebnis überraschte uns: Nicht die Politik-Hauptstadt Berlin, sondern die Finanzmetropole \textbf{Frankfurt am Main} führt das Ranking mit einem Durchschnittsbudget von ca. \textbf{199.679 €} an. Dicht darauf folgt \textbf{München} (189.133 €). Berlin landet mit ca. \textbf{183.790 €} nur auf dem dritten Platz, noch vor Hamburg (166.201 €). Dies zeigt deutlich, dass hohe Budgets vor allem im Finanz- und Industriesektor (Frankfurt/München) zu finden sind.

\begin{figure}[H]
\centering
\begin{tikzpicture}
\begin{axis}[
    ybar,
    ylabel={Ø Budget (EUR)},
    symbolic x coords={Hamburg, Berlin, München, Frankfurt},
    xtick=data,
    nodes near coords,
    nodes near coords align={vertical},
    width=0.9\textwidth,
    height=7cm,
    ymin=0,
    ymax=250,
    enlarge x limits=0.2,
    scaled y ticks=false,
    yticklabel={\pgfmathprintnumber[fixed,precision=0]{\tick}k\,€},
    nodes near coords={\pgfmathprintnumber[fixed,precision=0]{\pgfplotspointmeta}k\,€}
]
\addplot coordinates {
    (Hamburg,166)
    (Berlin,184)
    (München,189)
    (Frankfurt,200)
};
\end{axis}
\end{tikzpicture}
\caption{Vergleich der durchschnittlichen Lobby-Budgets deutscher Großstädte. Finanzmetropolen liegen vor der politischen Hauptstadt.}
\label{fig:cities}
\end{figure}

\chapter{Transparenz \& Compliance}

Ein Register steht und fällt mit der Datenqualität. Ein Teil unseres Dashboards widmet sich daher der "Meta-Analyse": Wie gut halten sich die Akteure an die Regeln?

\section{Die Verweigerer}
Das Gesetz erlaubt in bestimmten Härtefällen, Finanzdaten zu verweigern (Feld \texttt{refuse\_info}). Wir wollten wissen, wie oft von dieser Klausel Gebrauch gemacht wird.

\begin{figure}[H]
\caption{Ermittlung von Transparenz-Lücken}
\begin{lstlisting}[style=sql]
SELECT 
    COUNT(*) as "Anzahl Verweigerer",
    SUM(CASE WHEN refuse_reason IS NOT NULL THEN 1 ELSE 0 END) 
      as "Mit Begründung"
FROM financial_expenses
WHERE refuse_info = true;
\end{lstlisting}
\end{figure}

Diese Query speist ein "Alarm-Panel" im Dashboard. Aktuell haben \textbf{113 Akteure} die Finanzangaben verweigert, wovon nur 33 eine explizite Begründung hinterlegt haben. Dies ist zwar eine kleine Minderheit, aber für Journalisten ("Watchdogs") eine extrem relevante Zielgruppe.

\section{Heuristische Parteinähe-Erkennung}
Das Datenmodell enthält kein explizites Feld für "Parteizugehörigkeit". Dennoch möchten Nutzer wissen, welchem Spektrum ein Ex-Politiker zuzuordnen ist. Eine Text-Mining-Lösung in SQL analysiert deshalb das Feld \texttt{function\_position} (z.B. "Ehemaliger MdB (SPD)").

\begin{figure}[H]
\caption{SQL-basierte Textklassifikation von Parteizugehörigkeiten}
\begin{lstlisting}[style=sql]
SELECT 
  CASE 
    WHEN function_position ILIKE '%CDU%' 
      OR function_position ILIKE '%CSU%' THEN 'Union'
    WHEN function_position ILIKE '%SPD%' THEN 'SPD'
    WHEN function_position ILIKE '%Grüne%' THEN 'Grüne'
    WHEN function_position ILIKE '%FDP%' THEN 'FDP'
    ELSE 'Sonstige / Unbekannt'
  END as "Partei",
  COUNT(*) as "Treffer"
FROM recent_gov_house_reps
WHERE function_position IS NOT NULL
GROUP BY 1
ORDER BY 2 DESC;
\end{lstlisting}
\end{figure}

Dieser Ansatz demonstriert die Mächtigkeit von \texttt{CASE WHEN} Statements. Die Analyse ergab zwar eine Dominanz der Kategorie "Sonstige" (664 Treffer), was auf unstrukturierte Eingaben hindeutet. Unter den erkennbaren Parteien führte jedoch die \textbf{FDP (9)} vor Union (4) und SPD (4).

\begin{figure}[H]
\centering
\begin{tikzpicture}
\begin{axis}[
    ybar,
    ylabel={Anzahl Treffer},
    symbolic x coords={Grüne, SPD, Union, FDP},
    xtick=data,
    nodes near coords,
    width=0.8\textwidth,
    height=6cm,
    ymin=0,
    ymax=12
]
\addplot coordinates {
    (Grüne,2)
    (SPD,4)
    (Union,4)
    (FDP,9)
};
\end{axis}
\end{tikzpicture}
\caption{Erkannte Parteizugehörigkeiten bei Ex-Politikern (Text-Mining).}
\label{fig:parties}
\end{figure}

\section{Strukturanalyse: Rechtsformen und ihre Agenda}
Ein tieferer Blick in die Daten zeigt eine Arbeitsteilung zwischen den Rechtsformen. Untersucht wurde, welches Themenfeld bei welcher Rechtsform am häufigsten genannt wird und wie viel Budget durchschnittlich dahintersteht.

Die Analyse bestätigt das klassische Bild der Interessensvertretung:
\begin{itemize}
    \item \textbf{Zivilgesellschaft (e.V. und Stiftungen):} Hier dominieren ökologische Themen. Sowohl bei eingetragenen Vereinen als auch bei Stiftungen steht „Nachhaltigkeit und Ressourcenschutz“ unangefochten auf Platz 1.
    \item \textbf{Wirtschaft (GmbH und AG):} Kapitalgesellschaften setzen andere Prioritäten. Ihr Top-Thema ist konsistent „Wissenschaft, Forschung und Technologie“.
\end{itemize}

Dies deutet darauf hin, dass Unternehmen Lobbying primär als Hebel für In\-no\-va\-ti\-ons\-för\-de\-rung und Standortpolitik nutzen, während NGOs die normative Debatte um Nachhaltigkeit treiben.

\begin{table}[H]
\centering
\begin{tabular}{llc}
\toprule
\textbf{Rechtsform} & \textbf{Top-Thema} & \textbf{Ø Budget (EUR)} \\
\midrule
Stiftung & Nachhaltigkeit und Ressourcenschutz & 269.333 \\
Eingetragener Verein (e.V.) & Nachhaltigkeit und Ressourcenschutz & 262.830 \\
GmbH & Wissenschaft, Forschung und Technologie & 177.708 \\
Aktiengesellschaft (AG) & Wissenschaft, Forschung und Technologie & 94.242 \\
\bottomrule
\end{tabular}
\caption{Dominante Themenfelder je nach Rechtsform des Lobby-Akteurs.}
\end{table}

\chapter{Gesamtfazit}

Dieses Portfolio markiert den Abschluss eines umfassenden Data-Engineering-Projekts. Das Ziel, die Strukturen des Lobbyregisters sichtbar zu machen, wurde zu einer konsistenten Analytics-Plattform weiterentwickelt. Die Arbeit lässt sich in vier Phasen gliedern, die jeweils prägende Erkenntnisse hervorbrachten.

\section*{Phase 1: Modellierung}
Die Modellierung zeigte früh, wie komplex die Abbildung der Registerdaten ist. Die tief verschachtelten JSON-Strukturen der Bundestags-API erforderten ein streng normalisiertes Schema in der 3. Normalform mit über 80 Tabellen.
\\[4mm]
\textbf{Zentrale Erkenntnis:} 
\newline
Eine hohe Normalisierung sichert Datenintegrität und vermeidet Redundanzen, verschiebt aber Komplexität in spätere Performance-Optimierungen. Jedes Array in der Quelle benötigt eine eigene 1:N-Struktur – ein Aufwand, der sich zugunsten der Datenqualität auszahlt.

\section*{Phase 2: ETL Prozess}
In der Implementierungsphase stand Prozessstabilität im Mittelpunkt. Eine asynchrone Python-Pipeline entkoppelte Producer (API) und Consumer (Datenbank) und wurde komplett containerisiert.
\\[4mm]
\textbf{Zentrale Erkenntnis:} 
\newline
Resiliente ETL-Strecken entstehen durch saubere Entkopplung und reproduzierbare Umgebungen. \texttt{asyncio}, Queues und Docker sorgten dafür, dass die Pipeline auch unter Last kontrolliert arbeitet und teamfähig bleibt.

\section*{Phase 3: Optimierung}
Die anschließende Optimierung adressierte die unvermeidbare Latenz einer stark normalisierten Datenbank. Analysen mit vielen Joins wurden zu langsam, um sie direkt in Dashboards zu nutzen.
\\[4mm]
\textbf{Zentrale Erkenntnis:}
\newline
Performanz verlangt gezielte Redundanz. \textbf{Materialized Views} und spezialisierte \textbf{GIN-Indizes}\footnote{Generalized Inverted Index.} reduzierten Antwortzeiten von Sekunden auf Millisekunden und machten die Daten für interaktive Analysen nutzbar.

\section*{Phase 4: Visualisierung}
Die Visualisierung überführte die technische Basis in auswertbare Dashboards. Auf dieser Grundlage ließen sich Muster wie der Fokus auf das „Energieeffizienzgesetz“ oder PLZ-basierte Cluster transparent machen.
\\[4mm]
\textbf{Zentrale Erkenntnis:}
\newline
SQL bleibt auch im BI-Kontext ein zentrales Werkzeug. Erst die Kombination aus belastbarer Datenbasis und klaren Visualisierungen ermöglicht fundierte Aussagen zu Netzwerken, Budgets und Themenfeldern.

\section*{Schlusswort}
Das Modul „Advanced Database Techniques“ machte sichtbar, wie Modellierung, ETL, Optimierung und Visualisierung ineinandergreifen. Das Ergebnis ist eine nachvollziehbare Plattform, die mit modernen IT-Methoden Transparenz im politischen Umfeld fördert. Sie übergibt nicht nur Quellcode, sondern ein belastbares Werkzeug zur Analyse des Lobbyregisters.

%%%%%%%%%%%%%%%%%%%
%% Literatur
%%%%%%%%%%%%%%%%%%%
\backmatter
\nocite{*}
\printbibliography

\end{document}
